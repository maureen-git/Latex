\documentclass[11pt]{article}

%no indentation
\usepackage{parskip}

%afbeeldingen


%Accents and special characters
\usepackage[utf8]{inputenc}
\usepackage[T1]{fontenc}
\usepackage{gensymb}
\usepackage[version=4]{mhchem}

%equations, align left
% \usepackage[fleqn]{amsmath}

\usepackage[table,xcdraw]{xcolor}
%Dutch spelling
\usepackage[dutch]{babel}

%To force figures and tables in text with [H]
\usepackage{float}

%More space between equations
\setlength{\jot}{10pt}

%Text strikethrough
\usepackage{soul}

%Smaller margins
\usepackage[top=1in, bottom=1.25in, left=1.25in, right=1.25in]{geometry}

%tabel with color
\usepackage[table,xcdraw,dvipsnames]{xcolor}

\usepackage{mhchem}
\usepackage[hidelinks]{hyperref}
\usepackage{chemformula}
\usepackage{siunitx}
\DeclareSIUnit{\litre}{L}
\usepackage{enumitem}
\usepackage{derivative}
\usepackage{fancyhdr}
\usepackage{nccmath}
\usepackage{multirow}
\usepackage[dutch]{babel}
\usepackage{chemfig}
\begin{document}

\pagestyle{fancy}
\fancyhead{} % clear all header fields
\fancyhead{\textbf{IJkingstoets Chemische Reactorkunde} \hfill \textbf{30 oktober 2025}\\
	% \textbf{Prof. J. Degrève} \hfill }
	\textbf{Prof. F. Vermeire} \hfill \textbf{}}
\bigskip

\fancyfoot{} % clear all footer fields
\fancyfoot[R]{\thepage}
% % \fancyfoot[L]{\textbf{Schrijf op elk blad je naam.}\\
	% \textbf{Schrijf duidelijk! Wat niet leesbaar is, wordt niet verbeterd!}}

% 
% \textbf{Start het antwoord van elk deel op een nieuw blad.}

\textbf{Azijnzuur en vinyl acetaat monomeer (VAM) productie}
%\begin{figure}[h!]
    %\centering
%     \includegraphics[width=15cm]{Ijkingstoets_flowsheet_figure.svg.pdf}
%     \label{fig:temp_distribution}
% \end{figure}

Het Monsanto-proces wordt gebruikt voor de omzetting van methanol (\chemfig{CH_3OH}) en koolstofmonoxide (\chemfig{CO}) naar azijnzuur (\chemfig{CH_3COOH}). De reactie (reactie 1) vindt plaats in vloeistoffase in een CSTR. Na de reactie kan het product azijnzuur (\chemfig{CH_3COOH}) gescheiden worden van de overblijvende reagentia in een distillatiekolom. De overblijvende reagentia kunnen verder in het proces als inerte stoffen (I) worden beschouwd. De bodemstroom van de kolom bevat het product azijnzuur. Azijnzuur kan vervolgens gebruikt worden om, samen met ethyleen (\chemfig{C_2H_4}) en zuurstofgas (\chemfig{O_2}), vinyl acetaat monomeer (\chemfig{C_2H_3OOCCH_3}) te produceren. De reactie (reactie 2) vindt plaats in gasfase in een PFR. Alle reactoren mogen isotherm verondersteld worden. Noteer steeds alle assumpties die gemaakt worden.
\vspace{20pt} 
\begin{equation*}
\begin{split}
    \text{Reactie 1:} \quad & \ce{CH3OH + CO <=> CH3COOH} \quad \\
    &\text{met} \quad R_{\ce{CH3OH}} = -k_1 \quad [\text{mol/m}^3 \cdot \text{h}] \\
    \text{Reactie 2:} \quad & \ce{C2H4 + CH3COOH + 0.5 O2 -> C2H3OOCCH3 + H2O} \quad \\
    &\text{met} \quad R_{\ce{CH3COOH}} = -k_2 C_{\ce{CH3COOH}} \quad [\text{mol/m}^3 \cdot \text{h}]
\end{split}
\end{equation*}
\vspace{16pt}



\textbf{Vraag 1} \textbf{Bepaal het benodigde volume voor de CSTR}, indien de voedingstromen 1 en 2 bestaan uit, respectievelijk, zuiver methanol en zuiver koolstofmonoxide. Gebruik hiervoor de gegevens uit Tabel \ref{tabel 1}.
\\ \\
\noindent Massabalans over het mengpunt:\\
\\
$q_1* C_{\ce{CH3OH},1} \ - \ (q_1+q_2)*C_{\ce{CH3OH},3} \ = 0$\\
$C_{\ce{CH3OH},3} = 1.5 \ mol/m^3$\\
\\
$q_2* C_{\ce{CO},2} \ - \ (q_1+q_2)*C_{\ce{CO},3} \ = 0$\\
$C_{\ce{CO},3} = 1.0 \ mol/m^3$\\
\\
\\
\noindent Massabalans over CSTR\\
\\
$F_{\ce{CH3OH},3} - F_{\ce{CH3OH},4} + R_{\ce{CH3OH}}(C_{\ce{CH3OH},4})*V=0$\\
$(q_1+q_2)*C_{\ce{CH3OH},3} - (q_1+q_2)*C_{\ce{CH3OH},4} = k_1 * V$\\
$(q_1+q_2)*(C_{\ce{CH3OH},3} - C_{\ce{CH3OH},4} )= k_1 * V$\\
$(q_1+q_2)*(C_{\ce{CO},3} - C_{\ce{CO},4} )= k_1 * V$\\
$(q_1+q_2)*(C_{\ce{CO},3} - C_{\ce{CO},3} * (1-x_{\ce{CO}}))= k_1 * V$\\
$V = 81 m^3$\\

\vspace{16pt}
\begin{table}[h!]
\centering
\begin{tabular}{|l|l|l|}
\hline
\rowcolor[HTML]{EFEFEF} 
Parameter              & Symbool   & Waarde \\ \hline
Volumetrisch debiet stroom 1 & $q
_1$ & $3 \ m^3/h$   \\ \hline
Concentratie \ce{CH3OH} stroom 1 & $C_{\ce{CH3OH},1}$ & $4.5 \ mol/m^3$   \\ \hline
Volumetrisch debiet stroom 2 & $q
_2$ & $6 \ m^3/h$   \\ \hline
Concentratie \ce{CO} stroom 2 & $C_{\ce{CO},2}$ & $1.5 \ mol/m^3$   \\ \hline
Conversie \ce{CO} in CSTR     & $X_{\ce{CO}}$       & $0.9$  \\ \hline
Kinetische constante reactie 1  & $k_1$ & $0.1 \ mol/(m^3 . h)$  \\ \hline
\end{tabular}
\caption{Gegevens met betrekking tot de CSTR.}
\label{tabel 1}
\end{table}
\vspace{14pt}

\textbf{Vraag 2} Veronderstel dat al het azijnzuur aanwezig in de inputstroom van de distillatiekolom (stroom 4) uiteindelijk terecht komt in de bodemstroom van de distillatiekolom (stroom 6). Verder is er geweten dat de bodemstroom  uit 90 mol\% azijnzuur en 10 mol\% inerten bestaat. \textbf{Bereken het molaire debiet van azijnzuur en het molaire debiet van inerten in de bodemstroom van de distillatiekolom.}
\\ \\
\noindent Massabalans $\ce{CH3COOH}$ over de distillatiekolom:\\
\\
$(q_1+q_2)*C_{\ce{CH3COOH},4}-F_{\ce{CH3COOH}, 6}=0$\\
$(q_1+q_2)*(C_{\ce{CO},3}-C_{\ce{CO},4})-F_{\ce{CH3COOH}, 6}=0$\\
$(q_1+q_2)*(C_{\ce{CO},3}-C_{\ce{CO},3}*(1-x_{\ce{CO}}))-F_{\ce{CH3COOH}, 6}=0$\\
$F_{\ce{CH3COOH}, 6}=8.1 \ mol/h$\\
\\ \\
$F_{\ce{CH3COOH}, 6}=x_{\ce{CH3COOH},6}*F_{tot,6}$\\
$F_{tot,6}= 9\ mol/h$\\
\\
$F_{\ce{I}, 6}=x_{\ce{I},6}*F_{tot,6}$\\
$F_{I,6}= 0.9\ mol/h$\\
\\


\vspace{10pt}
\newpage
\textbf{Vraag 3} De bodemstroom van de distillatiekolom (stroom 6) wordt samengevoegd met een voedingsstroom bestaande uit \ce{C2H4} en \ce{O2} (stroom 7). De resulterende stroom (stroom 8) wordt gebruikt als inlaatstroom voor de PFR. \textbf{Bepaal het benodigde volume voor de PFR.} Gebruik hiervoor de bijkomende gegevens uit Tabel \ref{tabel 2}. Indien je vraag 2 niet hebt kunnen oplossen, mag je verder werken met $F_{\ce{CH3COOH},6}$ = $8.1\ mol/h$ en $F_{\ce{I},6}$ = $0.9 \ mol/h$.
\\ \\
\noindent Massabalans $\ce{CH3COOH}$ over de PFR:\\
\\
$F_{\ce{CH3COOH}}-(F_{\ce{CH3COOH}}+dF_{\ce{CH3COOH}})+R_{\ce{CH3COOH}}*dV=0$\\
$dF_{\ce{CH3COOH}}=R_{\ce{CH3COOH}}*dV$\\
$-F_{\ce{CH3COOH},8} *dx_{\ce{CH3COOH}}=-k_2 *C_{\ce{CH3COOH}}*dV$\\
$-F_{\ce{CH3COOH},8} *dx_{\ce{CH3COOH}}=-k_2 *C_{\ce{CH3COOH},8}*\frac{1-x_{\ce{CH3COOH}}}{1+\epsilon*x_{\ce{CH3COOH}}}*dV$\\
\\
\noindent Masssabalans integreren:\\
\\
%  $\int_{0}^{x_{\ce{CH3COOH}}}\frac{1+\epsilon*x_{\ce{CH3COOH}}}{{1-x_{\ce{CH3COOH}}}}*dx_{\ce{CH3COOH}}=\int_{0}^{V}\frac{k_2*C_{\ce{CH3COOH},8}}{F_{\ce{CH3COOH},8}}*dV$\\

%  $\int_{0}^{x_{\ce{CH3COOH}}}\frac{1}{{1-x_{\ce{CH3COOH}}}}dx_{\ce{CH3COOH}}+\int_{0}^{x_{\ce{CH3COOH}}}\frac{\epsilon*x_{\ce{CH3COOH}}}{{1-x_{\ce{CH3COOH}}}}dx_{\ce{CH3COOH}}=\int_{0}^{V}\frac{k_2*C_{\ce{CH3COOH},8}}{F_{\ce{CH3COOH},8}}dV$\\

%  $\int_{u_1}^{u_2}\frac{1}{u}*-du+\epsilon *\int_{u_1}^{u_2}\frac{1-u}{u}*-du=\frac{k_2*C_{\ce{CH3COOH},8}}{F_{\ce{CH3COOH},8}}\int_{0}^{V}dV$ met substitutie $u=1-x_{\ce{CH3COOH}}$\\

%  $-\bigl[ln(u)\bigr]_{u_1}^{u_2} - \epsilon \bigl[ln(u)-u\bigr]_{u_1}^{u_2}=\frac{k_2*C_{\ce{CH3COOH},8}}{F_{\ce{CH3COOH},8}}V$\\

%  $-\bigl[ln(u)\bigr]_{u_1}^{u_2} - \epsilon \bigl[ln(u)-u\bigr]_{u_1}^{u_2}=\frac{k_2*C_{\ce{CH3COOH},8}}{F_{\ce{CH3COOH},8}}V$\\

%  $\bigl[ln(1-x_{\ce{CH3COOH}})\bigr]_{x_{\ce{CH3COOH}}}^{0} + \epsilon \bigl[ln(1-x_{\ce{CH3COOH}})-(1-x_{\ce{CH3COOH}})\bigr]_{x_{\ce{CH3COOH}}}^{0}=\frac{k_2*C_{\ce{CH3COOH},8}}{F_{\ce{CH3COOH},8}}V$\\

%  $V=\frac{F_{\ce{CH3COOH},8}}{k_2*C_{\ce{CH3COOH},8}}*((1+\epsilon)*ln(\frac{1}{1-x_{\ce{CH3COOH}}})-\epsilon*x_{\ce{CH3COOH}})$\\
%  $V=\frac{8.1\frac{mol}{h}}{0.1\frac{1}{h}*8.5\frac{mol}{m^3}}((1+\epsilon)*ln(\frac{1}{}$\\
% % \\
% % \noindent Expansiefactor:
% % \\
% % $\epsilon = \frac{V_{x_{\ce{CH3COOH}}=1}-V_{x_{\ce{CH3COOH}}=0}}{V_{x_{\ce{CH3COOH}}=0}}$\\
% % $\epsilon= \frac{(0.9+8.1+8.1+5.95+1.9)-(0.9+8.1+10+10)}{0.9+8.1+10+10}=-0.14$\\
% % \\
% %\noindent Initiële concentratie:
% %\\
% %$C_{\ce{CH3COOH},8}=\frac{y_{\ce{CH3COOH},8}*P_8}{R*T_8}=\frac{0.279*1 \ atm}{0.08205 \frac{m^3*atm}{K*kmol}*400K}=8.5 \frac{mol}{m^3}$\\


\vspace{15pt}
\begin{table}[h!]
\centering
\begin{tabular}{|l|l|l|}
\hline
\rowcolor[HTML]{EFEFEF} 
Parameter              & Symbool   & Waarde \\ \hline
Molair debiet \ce{O2} stroom 7 & $F_{\ce{O2},7}$ & $10 \ mol/h$   \\ \hline
Molair debiet \ce{C2H4} stroom 7 & $F_{\ce{C2H4},7}$ & $10 \ mol/h$   \\ \hline
Conversie \ce{CH3COOH} in PFR     & $X_{\ce{CH3COOH}}$       & $0.9$  \\ \hline
Druk stroom 8  & $P_8$ & $1 \ atm$   \\ \hline
Temperatuur stroom 8 & $T_8$ & $400 \ K$   \\ \hline
Gasconstante & $R$ & $0.0821 \ (m^3 . atm)/(K.kmol)$   \\ \hline
Kinetische constante reactie 2  & $k_2$ & $0.1 \ h^{-1}$  \\ \hline
\end{tabular}
\caption{Gegevens met betrekking tot de PFR.}
\label{tabel 2}
\end{table}



\end{document}